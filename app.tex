\documentclass{beamer}
\usepackage[section]{placeins}
\usepackage{float}
\usepackage[position=b]{subfig}

\mode<presentation>{
  %   \usetheme{CambridgeUS}      % or try Darmstadt, Madrid, Warsaw, ...
  \usetheme{Frankfurt}
  % \definecolor{mygreen}{cmyk}{0.6,0.168,0.168,0.25}
  % \definecolor{mygreen}{RGB}{84,135,138}

  % \definecolor{myblue}{RGB}{23, 91, 127}
  \definecolor{myblue}{RGB}{24, 71, 160}
  \setbeamercolor{structure}{fg=myblue}
  \usefonttheme{default}
  \setbeamertemplate{navigation symbols}{}
  %\setbeamertemplate[compatibility=false, numbered]{caption}
  \setbeamertemplate{itemize items}[default]
  \setbeamertemplate{enumerate items}[default]
  \defbeamertemplate*{title page}{customized}[1][]{
  \centering
   \begin{beamercolorbox}[rounded=true,shadow=true,leftskip=1cm,colsep*=.75ex]{title}
  \centering
   \usebeamerfont{title}\inserttitle\par
    \end{beamercolorbox}
     \par\medskip\medskip\medskip
     \begin{columns}

     \begin{column}{0.6\textwidth}
        \usebeamerfont{author}\insertauthor\par\medskip\medskip
        \usebeamerfont{institute}\insertinstitute\par\medskip\medskip
        \usebeamerfont{date}\insertdate\par
        % \medskip \medskip \medskip
        \begin{figure}[!htbp]
         \begin{center}
          \subfloat{
            \includegraphics[width=0.4\columnwidth]{figuras/cnpqLogo.jpg}
          }
          \hspace{1cm}
          \subfloat{
            \includegraphics[width=0.2\columnwidth]{figuras/capesLogo.png}
          }
        \end{center}
        \end{figure}
      \end{column}
    \begin{column}{0.3\textwidth}
    \includegraphics[width=\columnwidth]{figuras/brasao_usp_pb}
    \end{column}
    \end{columns}
  }

  \setbeamertemplate{footline}{
    \begin{beamercolorbox}[colsep=1.5pt]{upper separation line foot}
    \end{beamercolorbox}
    \hbox{%
   \begin{beamercolorbox}[wd=0.82\paperwidth, ht=2.5ex, dp=1.125ex, left]{title in head/foot}%
        \usebeamerfont{title in head/foot}\hspace*{2ex}\insertshorttitle
      \end{beamercolorbox}%

      \begin{beamercolorbox}[wd=0.09\paperwidth, ht=2.5ex, dp=1.125ex, center]{title in head/foot}%
        \usebeamerfont{author in head/foot}{\insertshortinstitute }
      \end{beamercolorbox}%

      \begin{beamercolorbox}[wd=0.09\paperwidth, ht=2.5ex, dp=1.125ex, right]{title in head/foot}%
        \usebeamerfont{title in head/foot}\insertframenumber/\inserttotalframenumber\hspace*{2ex}
      \end{beamercolorbox}}
    \begin{beamercolorbox}[colsep=1.5pt]{lower separation line foot}
    \end{beamercolorbox}
  }
}

\makeatletter
\renewcommand{\itemize}[1][]{%
  \beamer@ifempty{#1}{}{\def\beamer@defaultospec{#1}}%
  \ifnum \@itemdepth >2\relax\@toodeep\else
    \advance\@itemdepth\@ne
    \beamer@computepref\@itemdepth% sets \beameritemnestingprefix
    \usebeamerfont{itemize/enumerate \beameritemnestingprefix body}%
    \usebeamercolor[fg]{itemize/enumerate \beameritemnestingprefix body}%
    \usebeamertemplate{itemize/enumerate \beameritemnestingprefix body begin}%
    \list
      {\usebeamertemplate{itemize \beameritemnestingprefix item}}
      {\def\makelabel##1{%
          {%
            \hss\llap{{%
                \usebeamerfont*{itemize \beameritemnestingprefix item}%
                \usebeamercolor[fg]{itemize \beameritemnestingprefix item}##1}}%
          }%
        }%
      }
  \fi%
  \beamer@cramped%
  \justifying% NEW
  %\raggedright% ORIGINAL
  \beamer@firstlineitemizeunskip%
}
\makeatother

\useoutertheme[subsection=false]{miniframes}
\usepackage{etoolbox}
\makeatletter
\patchcmd{\slideentry}{\advance\beamer@xpos by1\relax}{}{}{}
\def\beamer@subsectionentry#1#2#3#4#5{\advance\beamer@xpos by1\relax}%
\makeatother

\usepackage{ragged2e}
\usepackage{lipsum}
\usepackage[sort, numbers]{natbib}
% \usepackage[style=alphabetic]{biblatex}

\usepackage[linesnumbered,ruled,vlined]{algorithm2e}
%\usepackage{algorithm}
\usepackage{algpseudocode}
\SetKwFor{Para}{para}{fa\c{c}a}{fim}
\SetKwFor{ParaCada}{para cada}{fa\c{c}a}{fim}
\SetKwRepeat{DoWhile}{faça}{enquanto}

\usepackage[compatibility=false]{caption}
% \usepackage{subcaption}
\usepackage[brazil]{babel}
\usepackage[utf8x]{inputenc}
\usepackage{setspace}
\usepackage{ragged2e}
\usepackage{graphicx}
\usepackage{amsmath,amsfonts,amssymb}
\usepackage{xcolor}
\usepackage{url}
\usepackage{array}
\usepackage{gensymb}
\usepackage{multicol}
\usepackage[loadonly]{enumitem}
\newlist{arrowlist}{itemize}{1}
\setlist[arrowlist]{label=$\Rightarrow$}
\usepackage{listings}
\newcommand{\fonte}[1]{\caption*{Fonte: #1}}
\newcommand{\fonteminha}{\caption*{Fonte: Elaborado pelo autor.}}

\title[\textbf{Geração de imagens artificiais e quantização aplicadas a problemas de classificação}]{\textbf{Geração de imagens artificiais e quantização aplicadas a problemas de classificação}}
\author{Gabriela Salvador Thumé \\ \vspace{4pt}
        \tiny Orientador: Prof. Dr. Moacir Antonelli Ponti \\ \vspace{4pt}
        \tiny Co-orientador: Prof. Dr. João do Espirito Santo Batista Neto}
\institute[ICMC/USP]{Instituto de Ciências Matemáticas e de Computação \\
Universidade de São Paulo \\ }
\date{29 de abril de 2016}
%%%%%%%%%%%%%%%%%%%%%%%%%%%%%%%%%%%%%%%%%%%%%%%%%%%%%%%%%%%%%%%%%%%%%%%%%%%%%%%
\begin{document}
\setbeamercovered{transparent}
\begin{frame}[plain]
  \maketitle
\end{frame}
%%%%%%%%%%%%%%%%%%%%%%%%%%%%%%%%%%%%%%%%%%%%%%%%%%%%%%%%%%%%%%%%%%%%%%%%%%%%%%%
\begin{frame}[noframenumbering]{Estrutura da Apresentação}
\setstretch{1.2}
% \begin{multicols}{2}
  \tableofcontents
% \end{multicols}
\end{frame}
\section{Introdução}
%%%%%%%%%%%%%%%%%%%%%%%%%%%%%%%%%%%%%%%%%%%%%%%%%%%%%%%%%%%%%%%%%%%%%%%%%%%%%%%
\AtBeginSection[]{
\begin{frame}<beamer>[noframenumbering]{Estrutura da Apresentação}
\setstretch{1.2}
% \begin{multicols}{2}
\tableofcontents[currentsection, sectionstyle=show/shaded,
]
% \tableofcontents[currentsection,currentsubsection, hideothersubsections,
%     sectionstyle=show/shaded,
% ]
% \end{multicols}
\end{frame}
}
%%%%%%%%%%%%%%%%%%%%%%%%%%%%%%%%%%%%%%%%%%%%%%%%%%%%%%%%%%%%%%%%%%%%%%%%%%%%%%%
\begin{frame}{Introdução}
\setstretch{1.2}
\setlength\leftmargini{0em}
\justifying
\begin{itemize}
\item Classificação de imagens;
\pause
\item Extração de características;
\pause
\item Algoritmos de aprendizado de máquina: modelo de representação;
\item Generalização permite classificar novos exemplos;
\pause
\item Características que não são suficientes para a diferenciação entre as classes;
\item Encontrar as características que melhor discriminam as classes.
\end{itemize}
\end{frame}
%-----------------------------------------------------------------------------
\subsection{Motivação}
\begin{frame}[plain]{Motivação}
\begin{figure}
    \includegraphics[height=0.9\textheight]{figuras/flow.png}
    \caption{Etapas canônicas do reconhecimento de padrões.}
\end{figure}
\end{frame}
%-----------------------------------------------------------------------------
\begin{frame}{Motivação}
\setstretch{1.2}
\setlength\leftmargini{0em}
\justifying
\begin{itemize}
\item Maior esforço ao operar no espaço de características já obtidas;
\item Transformações do espaço ou sistemas complexos de classificação para lidar com as deficiências das características extraídas;
\item Características que podem ser exploradas além dos métodos clássicos;
\item Investigar métodos de processamento e preparação de imagens antes da extração.
\end{itemize}
\end{frame}
%-----------------------------------------------------------------------------
\begin{frame}{Motivação - Características Latentes}
\setlength\leftmargini{0em}
\justifying
\setstretch{1.2}
\begin{itemize}
\item Justificado o uso de métodos de processamento e preparação de imagens antes da extração;
\item Podem revelar características latentes, que possam melhor descrever certas classes, utilizando algoritmos sobre as imagens originais.
\end{itemize}
\end{frame}
%-----------------------------------------------------------------------------
\begin{frame}{Motivação}
\setlength\leftmargini{0em}
\justifying
 \setstretch{1.2}
  \begin{itemize}
\justifying
    \item 98\% de acurácia após pré-processamento e segmentação (Rocha et al., 2010); % frutas
    \item Quantização pode impactar a classificação (Kanan e Cottrell, 2012);
    \item Quantização permite obter vetores de características mais compactos e com maior capacidade de discriminação entre classes (Ponti et al., 2014);
  \end{itemize}
\end{frame}
%-----------------------------------------------------------------------------
\begin{frame}{Motivação - Desbalanceamento de classes}
\setlength\leftmargini{0em}
\justifying
% \setstretch{1.2}
  \begin{itemize}
    \item Diferença entre o número de exemplos disponíveis;
    \item Imagens representam eventos importantes mas menos frequentes;
    \item Obstáculo, métodos de transformação do espaço e de
    classificação assumem que a base está balanceada;
    \item Foco: \emph{geração de imagens artificiais a partir do processamento de características das imagens da classe minoritária.}
  \end{itemize}
  \begin{figure}[htbp]
 \begin{center}
   \includegraphics[width=.4\linewidth]{figuras/imagemgerada.jpg}
 \caption{Imagem artificialmente gerada.}
 \end{center}
\end{figure}

\end{frame}
%-----------------------------------------------------------------------------
\subsection{Hipóteses}
\setstretch{1.2}
\setlength\leftmargini{1em}
\justifying
 \begin{frame}{Hipóteses}
  \begin{block}{Utilizar um número reduzido de cores}
    \justifying
    \begin{itemize}
      \item Juntamente com um método de quantização apropriado;
      \item Antes da extração de características;
      \item \textit{Pode permitir obter vetores de características mais compactos e com maior capacidade de discriminação entre classes.}
    \end{itemize}
  \end{block}
  \begin{block}{Geração de imagens artificiais}
    \justifying
    \begin{itemize}
      \item Balancear as classes;
      \item Preparação para a extração de características;
      \item \textit{Melhorar a acurácia, quando comparada à geração de exemplos artificiais no espaço de atributos.}
    \end{itemize}
  \end{block}
\end{frame}
%%%%%%%%%%%%%%%%%%%%%%%%%%%%%%%%%%%%%%%%%%%%%%%%%%%%%%%%%%%%%%%%%%%%%%%%%%%%%%%
\subsection{Contribuições}
\begin{frame}{Contribuição geral}
\setstretch{1.2}
\setlength\leftmargini{1em}
\justifying
  \begin{block}{}
  \justifying
  \emph{Investigar os métodos de pré-processamento para preparar uma coleção de imagens para a extração de características.}

  \vspace{5mm}
  Observa-se o efeito da quantização de imagens e do balanceamento do número de instâncias de diferentes classes na classificação.
  \end{block}
\end{frame}
%-----------------------------------------------------------------------------
\begin{frame}{Contribuições específicas}
% \setstretch{1.2}
\setlength\leftmargini{1em}
\justifying
    \begin{itemize}
      \item Demostrar que é possível obter vetores de características compactos e efetivos ao extrair imagens quantizadas;
      \begin{itemize}
        \item Custo computacional baixo;
        \item Reduzindo o tamanho do vetor após a quantização e posterior extração de características;
        \item Redução do tempo de processamento para os métodos de descrição de textura.
      \end{itemize}
      \pause
      \item Demostrar que a geração de imagens artificiais pode contribuir com o balanceamento entre classes.
        \begin{itemize}
          \item Melhorando o \textit{F1-Score} resultante de algoritmos de classificação;
          \item Comparando com a geração de exemplos artificiais no espaço de atributos (SMOTE) e à classificação da base original.
        \end{itemize}
    \end{itemize}
\end{frame}
%-----------------------------------------------------------------------------
\begin{frame}{Contribuições em código e reprodutibilidade}
  \setstretch{1.2}
  \setlength\leftmargini{1em}
  \justifying
  \begin{itemize}
    \item Código para a quantização: \url{http://dx.doi.org/10.5281/zenodo.15932}
    \item Código da geração artificial: \url{https://github.com/GabiThume/msc-src}
  \end{itemize}
\end{frame}
%-----------------------------------------------------------------------------
\begin{frame}{Estrutura}
\begin{figure}
    \includegraphics[height=0.75\textheight]{figuras/geral.pdf}
    \caption{Estrutura geral desta pesquisa.}
\end{figure}
\end{frame}
%-----------------------------------------------------------------------------
\section{Contextualização}
\subsection{Pré-processamento}
\begin{frame}{Pré-processamento de Imagens}
\begin{figure}
    \includegraphics[height=0.75\textheight]{figuras/geral_metodos.pdf}
\end{figure}
\end{frame}
%-----------------------------------------------------------------------------
\begin{frame}{Pré-processamento de Imagens}
\begin{figure}[htbp]
 \begin{center}
   \includegraphics[width=1\linewidth]{figuras/preprocessamento.png}
 \caption{Conversão em escala de cinza, borramento, realce e de equalização de histograma.}
 \end{center}
\end{figure}
\end{frame}
%-----------------------------------------------------------------------------
\begin{frame}{Pré-processamento de Imagens - Convolução}
\setlength\leftmargini{0em}
\justifying
\setstretch{1.2}
\begin{itemize}
    \item Percorre a imagem com um filtro espacial rotacionado em $180\degree$; \\
    \item Cria cada novo pixel com as mesmas coordenadas do centro da vizinhança contendo o valor resultante da filtragem.
\end{itemize}
\begin{figure}[htbp]
 \begin{center}
   \includegraphics[width=.5\linewidth]{figuras/convolucao.png}
 \caption{Convolução com filtro previamente rotacionado.}
 \end{center}
\end{figure}
\end{frame}
%-----------------------------------------------------------------------------
\begin{frame}{Pré-processamento de Imagens - Convolução}
\begin{figure}[!htbp]
  \begin{center}
  \subfloat[Original]{
    \includegraphics[width=0.5\linewidth]{figuras/original.png}
  }
  \subfloat[Filtragem Gaussiana]{
    \includegraphics[width=0.5\linewidth]{figuras/blur.png}
  }
  \end{center}
\end{figure}
\end{frame}
%-----------------------------------------------------------------------------
\begin{frame}{Pré-processamento de Imagens - Realce}
\begin{figure}[!htbp]
  \begin{center}
  \subfloat[Original]{
    \includegraphics[width=0.5\linewidth]{figuras/original.png}
  }
  \subfloat[Unsharp masking]{
    \includegraphics[width=0.5\linewidth]{figuras/unsharpmask.png}
  }
 \end{center}
\end{figure}
\end{frame}
%-----------------------------------------------------------------------------
\begin{frame}{Pré-processamento de Imagens - Quantização}
\begin{figure}[!htbp]
 \begin{center}
  \subfloat[Original]{
    \includegraphics[width=.22\linewidth]{\detokenize{figuras/quantization/Lenna.png}}
  }
  \subfloat[Intensidade']{
    \includegraphics[width=.22\linewidth]{\detokenize{figuras/quantization/Intensity_gray.png}}
    \label{fig:intensidade}
  }
  \subfloat[Gleam]{
    \includegraphics[width=.22\linewidth]{\detokenize{figuras/quantization/Gleam_gray.png}}
    \label{fig:gleam}
  }
  \newline
  \subfloat[Luminância']{
    \includegraphics[width=.22\linewidth]{\detokenize{figuras/quantization/Luminance_gray.png}}
    \label{fig:luminance}
  }
  \subfloat[Luma]{
    \includegraphics[width=.22\linewidth]{\detokenize{figuras/quantization/Luma_gray.png}}
    \label{fig:luma}
  }
  \subfloat[MSB]{
    \includegraphics[width=.22\linewidth]{\detokenize{figuras/quantization/MSB_gray.png}}
    \label{fig:msb}
  }
  \end{center}
  % \caption{Conversão para a escala de cinza com os métodos utilizados nessa pesquisa. Os métodos resultam em uma imagem com 8 bits (256 cores).}
  \end{figure}
\end{frame}
%-----------------------------------------------------------------------------
\subsection{Extração de características}
\begin{frame}{Extração de Características}
\begin{figure}
    \includegraphics[height=0.75\textheight]{figuras/geral_extracao.pdf}
\end{figure}
\end{frame}
%-----------------------------------------------------------------------------
\begin{frame}{Extração de Características}
\setlength\leftmargini{0em}
\justifying
\begin{itemize}
\item Descrever as informações visuais relevantes em um vetor de características;
\item Entrada para o classificador de padrões;
% Exemplo: importante para a discriminação entre classes de algas é a forma.
\item Salientar as diferenças entre imagens de classes distintas e suavizar possíveis diferenças de imagens da mesma classe (Ex. algas - forma).
\end{itemize}
\setlength\leftmargini{0em}
\begin{description}
\item [Textura:] suavidade, aspereza e uniformidade. Ex. entropia;
\item [Forma:] características externas. Ex. curvatura;
\item [Cor:] distribuição espacial de cores na imagem. Ex. histograma.
\end{description}
\end{frame}
%-----------------------------------------------------------------------------
\subsection{Desbalanceamento de classes}
\begin{frame}{Desbalanceamento de classes}
\begin{figure}
    \includegraphics[height=0.75\textheight]{figuras/geral_geracao.pdf}
\end{figure}
\end{frame}
%-----------------------------------------------------------------------------
\begin{frame}{Desbalanceamento de classes}
\setstretch{1.2}
\setlength\leftmargini{0em}
\justifying
\setstretch{1.2}
\begin{itemize}
  \item Número desbalanceado de exemplos. Majoritárias x minoritárias.
  \item Abordagens:
    \begin{itemize}
        \item \emph{Modificar métodos de aprendizagem:} adicionar funções de custo na classificação;
        \item \emph{Pré-processamento ao reamostrar os dados:}
      \begin{itemize}
          \item Aumentar a minoritária;
          \item Diminuir a majoritária.
      \end{itemize}
    \end{itemize}
\end{itemize}
\end{frame}
%-----------------------------------------------------------------------------
\begin{frame}{Desbalanceamento de classes - Subamostragem}
\setlength\leftmargini{0em}
\justifying
    \begin{itemize}
        \item Diminuir o número de elementos do conjunto;
        \item Podem remover informações essenciais dos dados originais;
        \item Eliminar elementos distantes da fronteira de decisão;
        \item Normalmente apresentam resultados piores.
        % \item Não há melhor para todos os cenários.
    \end{itemize}
\end{frame}
%-----------------------------------------------------------------------------
\begin{frame}{Desbalanceamento de classes - Sobreamostragem}
\setlength\leftmargini{0em}
\justifying
    \begin{itemize}
        \item Aumentar o número de elementos;
    \begin{block}{SMOTE}
\setlength\leftmargini{1em}
        \begin{itemize}
            \item Multiplica a diferença entre o vetor de características de um elemento e do seu vizinho mais próximo por um número $0 \leq x \leq 1$;
            \item Adiciona ao vetor original, criando um novo elemento entre os dois vetores originais;
            \item Aprendido como exemplo da classe minoritária;
            \item Força uma região de decisão maior e mais geral;
        \end{itemize}
    \end{block}
    \end{itemize}
\end{frame}
%-----------------------------------------------------------------------------
\begin{frame}{Desbalanceamento de classes - Sobreamostragem}
\setlength\leftmargini{0em}
\justifying
  \begin{algorithm}[H]
    \caption{SMOTE: método para rebalancear classes}
    \SetAlgoLined
    \Entrada{Imagem colorida $I$ em formato RGB}
    \Saida{Exemplos sintéticos $S$}
    $N \gets \textit{vizinhos(classe minoritária)}$\;
    \ParaCada{exemplo da classe minoritária}{
      $nn \gets \textit{vizinho aleatório de N}$\;
      $\textit{novo\_elemento} \gets \emptyset $\;

      \ParaCada{\textit{característica} $(x,y)$ do exemplo}{
        $\textit{diferença} \gets nn(x,y) - exemplo(x,y)$\;
        $\textit{gap} \gets \textit{número aleatório entre 0 e 1}$\;
        $\textit{novo\_elemento}(x,y) \gets exemplo(x,y) + gap*\textit{diferença}$\;
      }
      $S \gets S \cup \textit{novo\_elemento}$\;
    }
  \end{algorithm}
\end{frame}
%-----------------------------------------------------------------------------
\begin{frame}{Desbalanceamento de classes - SMOTE}
\setlength\leftmargini{0em}
\justifying
 \begin{figure}[hbpt]
 \begin{center}
   \includegraphics[width=.6\linewidth]{{figuras/smote}}
 \end{center}
\end{figure}
\justifying
    \begin{itemize}
        \item Rebalancear ao gerar novos elementos, ao invés de replicá-los;
        \item Sobre os vetores de características previamente extraídos;
        \item (Chawla et al., 2002) \textbf{Diferentes estratégias para criar exemplos sintéticos podem melhorar a performance da classificação;}
        \item Utilizado para comparação.
    \end{itemize}
\end{frame}
%-----------------------------------------------------------------------------
\begin{frame}{Redução de dimensionalidade}
  Precisa?
\end{frame}
%%%%%%%%%%%%%%%%%%%%%%%%%%%%%%%%%%%%%%%%%%%%%%%%%%%%%%%%%%%%%%%%%%%%%%%%%%%%%%%
\section{Quantização de imagens}
\begin{frame}{Quantização de imagens}
\setstretch{1.2}
\setlength\leftmargini{1em}
\begin{block}{}
\justifying
  \begin{itemize}
  \item O pipeline de reconhecimento de imagens inclui a conversão de imagens coloridas em imagens com apenas um canal ($2^3$ = 8 bits, $2^8$ = 256 intensidades);
  \item Após, essas imagens quantizadas são processadas por métodos de extração de características;
  \item Essa pesquisa explorou essa etapa para produzir vetores mais compactos.
  \item Diferentes parâmetros de quantização combinados com quatro métodos de extração de cor e um de textura.
  \end{itemize}
\end{block}
\end{frame}
%-----------------------------------------------------------------------------
\begin{frame}{Quantização de imagens}
  \begin{figure}
    \begin{center}
      \centering
      \subfloat[Original]{
        \includegraphics[width=.2\linewidth]{\detokenize{figuras/quantization/fig_quanttest.png}}
      }
      \subfloat[Gleam]{
        \includegraphics[width=.2\linewidth]{\detokenize{figuras/quantization/fig_quantGleam.png}}
      }
      \subfloat[Intensidade']{
        \includegraphics[width=.2\linewidth]{\detokenize{figuras/quantization/fig_quantIntensity.png}}
      }
      \subfloat[Luminância']{
        \includegraphics[width=.2\linewidth]{\detokenize{figuras/quantization/fig_quantLuminance.png}}
      }
      \subfloat[MSB]{
        \includegraphics[width=.2\linewidth]{\detokenize{figuras/quantization/fig_quantMSB.png}}
      }
    \end{center}
    \caption{A imagem original resultou em versões de um canal de cor com 232 intensidades únicas para o método (e) MSB e 184 intensidades para os demais métodos. Observa-se que os métodos\emph{Luminância'} e MSB conseguiram uma melhor discriminação entre cores.}
  \end{figure}
\end{frame}
%-----------------------------------------------------------------------------
\begin{frame}{Quantização de imagens}
  \begin{figure}
    \begin{center}
      \centering
      \includegraphics[width=\linewidth]{\detokenize{figuras/quantization/fig_quantizationexample.jpg}}
    \end{center}
    \caption{Exemplo de duas imagens da base de dados \emph{Caltech101-600} com variações no parâmetro de cor utilizando o método MSB. Da esquerda para a direita: imagem original 24-bits e suas versões quantizadas com: 256, 64, 32, 16 e 8 cores.}
  \end{figure}
\end{frame}
%-----------------------------------------------------------------------------
\subsection{Experimentos}
\begin{frame}{Experimentos}
\setstretch{1.2}
\setlength\leftmargini{1em}
\begin{block}{}
\justifying
  \begin{itemize}
  \item Experimentos utilizando um método de extração de características seguido pela classificação (sem posterior seleção de características);
  \item Experimentos utilizando o vetor resultante da concatenação de todos os métodos de extração, seguido pela classificação com e sem a seleção de características.
  \end{itemize}
\end{block}
\end{frame}
%-----------------------------------------------------------------------------
\begin{frame}{Experimentos}
  \begin{figure}
    \begin{center}
      \centering
      \includegraphics[width=0.4\linewidth]{\detokenize{figuras/quantization/quantizationResult.pdf}}
    \end{center}
    \caption{Fluxo das operações e os métodos utilizados nos experimentos.}
  %  Após a aquisição da imagem, ela é convertida para escala de cinza e seus níveis de cor são reduzidos de acordo com um parâmetro da quantização (i.e.\ número de cores). Dependendo do método, a correção \emph{gamma} é realizada. A imagem quantizada serve então como entrada para um método de extração de características e posteriormente é classificada com \emph{SVM}. Uma das etapas de experimentos prevê também a concatenação de todos os vetores extraídos e a seleção das características com \emph{LPP} antes da classificação.}
    \label{fig:quant:flowResult}
  \end{figure}
\end{frame}
%-----------------------------------------------------------------------------
\begin{frame}{Experimentos - Bases de Imagens}
  \begin{figure}[!htbp]
    \begin{center}
      \subfloat[Base de imagens Corel-1000]{
        \includegraphics[width=\linewidth]{\detokenize{figuras/quantization/fig_COREL_dataset.jpg}}
      }
      \newline
      \subfloat[Base de imagens Caltech101-600]{
        \includegraphics[width=\linewidth]{\detokenize{figuras/quantization/fig_Caltech101_dataset.jpg}}
      }
      \newline
      \subfloat[Base de imagens Produce]{
        \includegraphics[width=\linewidth]{\detokenize{figuras/quantization/fig_Produce_dataset.jpg}}
      }
    \end{center}
    \caption{Bases de imagens Corel-1000, Caltech101-600 e Produce. Elas são utilizadas nos experimentos de quantização.}
  \end{figure}
\end{frame}
%-----------------------------------------------------------------------------
\begin{frame}{Experimentos --- Protocolo}
\setstretch{1.2}
\setlength\leftmargini{1em}
\begin{block}{}
\justifying
\begin{enumerate}
  \item \textbf{Quantização}: com os métodos \emph{Intensidade'}, \emph{Gleam}, \emph{Luminância'} e MSB.
  \item \textbf{Extração de características}: utilizando os métodos e parâmetros escolhidos com base nas recomendações dos artigos que proporam tais métodos:
  \begin{itemize}
    \item \textit{Auto Color Correlogram} (ACC): a métrica de distância utilizada entre os pixels $p(x,y)$ e $q(s,t)$ é a tabuleiro de xadrez $D_8(p,q) = Max(|x-s| + d, |y-t| + d)$ para quatro distâncias $d =$ 1, 3, 5 e 7;
    \item \textit{Border-Interior Classification} (BIC): com uma vizinhança de quatro pixels;
    \item \textit{Color Coherence Vector} (CCV): adotando um valor de $\mathit{threshold} = 25$ para a classificação dos pixels entre coerentes e incoerentes;
    \item Haralick-6: o pixel vizinho para o qual iniciar a computar a matriz de co-ocorrência foi o pixel à direita.
  \end{itemize}
  \item \textbf{Redução da dimensionalidade}: a projeção utilizando \textit{Locality Preserving Projections} (LPP) foi realizada com o parâmetro $k =$ 128, 64, 32 e 16 dimensões e 10 vizinhos. Esse parâmetro foi determinado empiricamente e não influencia consideravelmente a acurácia.
  \item \textbf{Classificação}: realizada com o classificador \textit{Support Vector Machines} (SVM). Os parâmetros para essa etapa foram encontrados utilizando uma \textit{grid search} no conjunto de treinamento.
\end{enumerate}
\end{block}
\end{frame}
%-----------------------------------------------------------------------------
\begin{frame}{Experimentos --- Resultados}
\setstretch{1.2}
\setlength\leftmargini{1em}
\begin{block}{}
\justifying
\begin{enumerate}
\item
\end{enumerate}
\end{block}
\end{frame}
%-----------------------------------------------------------------------------
\begin{frame}{Experimentos --- Resultados}
\begin{figure}[!htbp]
  \begin{center}
    \centering
    \includegraphics[width=\linewidth]{\detokenize{figuras/quantization/fig_results_individual.png}}
  \end{center}
  \caption{Resultados para as bases Corel-1000 (a), Produce (b) e Caltech101-600 (c), utilizando todos os métodos de quantização. Para cada método de extração de características a acurácia é resultante da sua aplicação utilizando 256, 128, 64, 32, 16 e 8 cores, da esquerda para a direita.}
\end{figure}
\end{frame}
%-----------------------------------------------------------------------------
\begin{frame}{Experimentos --- Resultados}
  \begin{figure}[!htbp]
    \begin{center}
      \centering
      \includegraphics[width=\linewidth]{\detokenize{figuras/quantization/fig_results_individual_boxplotBIC.png}}
    \end{center}
    \caption{Resultados de acurácia média da classificação utilizando o método de quantização MSB considerando 256, 128, 64 e 32 cores com o método de extração de características BIC. Os  \textit{boxplots} em cinza correspondem às significâncias estatísticas com $\textit{p-value} < 0.01$ quando comparado à acurácia de 256 cores.}
  \end{figure}
\end{frame}
%-----------------------------------------------------------------------------
\begin{frame}{Experimentos --- Resultados}
  \begin{figure}[!htbp]
    \begin{center}
      \centering
      \includegraphics[width=\linewidth]{\detokenize{figuras/quantization/fig_results_individual_boxplotHaralick.png}}
    \end{center}
    \caption{Acurácia média da classificação após a utilização do método de quantização \emph{Luminância'} considerando 256, 128, 64 e 32 cores com o descritor Haralick. Os \textit{boxplots} em cinza correspondem às significâncias estatísticas com $\textit{p-value} < 0.01$ quando comparado à acurácia de 256 cores}
  \end{figure}
\end{frame}
%-----------------------------------------------------------------------------
\begin{frame}{Experimentos --- Resultados}
  \begin{figure}[!htbp]
    \begin{center}
      \centering
      \includegraphics[width=\linewidth]{\detokenize{figuras/quantization/fig_results_individual_boxplotMSBLPP.png}}
    \end{center}
    \caption{Resultados de acurácia para os método MSB (quantização), LPP (redução de dimensionalidade) e BIC (extração de características). A comparação do LPP versus MSB foi realizada com a mesma dimensionalidade. Os boxplots em cinza correspondem às significâncias estatísticas com $p < 0.01$ quando comparado a acurácia de 256 cores.}
  \end{figure}
\end{frame}
%-----------------------------------------------------------------------------
\begin{frame}{Experimentos --- Resultados}
  \begin{figure}[!htbp]
    \begin{center}
      \centering
      \includegraphics[width=\linewidth]{\detokenize{figuras/quantization/fig_results_full.png}}
    \end{center}
    \caption{Comparação da acurácia alcançada com diferentes métodos de quantização: \emph{Gleam}, \emph{Intensidade'}, \emph{Luminância'} e MSB. Inicialmente as imagens foram convertidas para escala de cinza com esses quatro métodos e foram dadas como entrada para todos os métodos de extração. O vetor de características resultante com $D=2310$ sofreu então redução da dimensionalidade com o método LPP para $d = 1160$, $582$, $294$ e $150$.}
  \end{figure}
\end{frame}
%-----------------------------------------------------------------------------
\begin{frame}{Experimentos --- Resultados}
  \begin{figure}[!htbp]
    \begin{center}
      \centering
      \includegraphics[width=\linewidth]{\detokenize{figuras/quantization/fig_results_full_boxplot.png}}
    \end{center}
    \caption{Comparação da acurácia com o uso da projeção LPP e o método MSB para quantização das imagens com o objetivo de redução de dimensionalidade.}
  \end{figure}
\end{frame}
%-----------------------------------------------------------------------------
\begin{frame}{Experimentos --- Resultados}
  \begin{figure}[!htbp]
    \begin{center}
      \centering
      \includegraphics[width=0.6\linewidth]{\detokenize{figuras/quantization/fig_results_full_LPP}}
    \end{center}
    \caption{Resultados para a projeção do LPP sobre o espaço de características produzido pelo método de quantização MSB utilizando 256 ($d = 2310$) e 64 cores ($d=582$).}
  \end{figure}
\end{frame}
%-----------------------------------------------------------------------------
\begin{frame}{Experimentos --- Discussão}
\setstretch{1.2}
\setlength\leftmargini{1em}
\begin{block}{}
\justifying
\begin{enumerate}
\item Aplicar a quantização na etapa de pré-processamento causa a redução da dimensionalidade do vetor de características no início do pipeline, beneficiando todas as etapas posteriores;
\item Utilizar um número reduzido de cores pode reduzir significativamente a dimensionalidade, enquanto melhora ou mantém a classificação do sistema;
\item Ao comparar o uso da quantização com a utilização de métodos mais complexos para a redução da dimensionalidade, esse processamento permite uma redução significante, enquanto normalmente preserva ou melhora a acurácia do sistema.
\end{enumerate}
\end{block}
\end{frame}
%%%%%%%%%%%%%%%%%%%%%%%%%%%%%%%%%%%%%%%%%%%%%%%%%%%%%%%%%%%%%%%%%%%%%%%%%%%%%%%
\section{Geração de imagens artificiais}
\begin{frame}{Geração de imagens artificiais}
\setstretch{1.2}
\setlength\leftmargini{1em}
\begin{block}{}
\justifying
\begin{itemize}
\item Compensar a baixa disponibilidade de exemplos de uma determinada classe;
\item Permitir a extração de informações antes não disponíveis nas imagens originais por meio da combinação ou perturbação das imagens de entrada.
\end{itemize}
\end{block}
\end{frame}
%-----------------------------------------------------------------------------
\begin{frame}{Geração de imagens artificiais}
  \begin{figure}
    \begin{center}
      \includegraphics[width=0.7\linewidth]{\detokenize {figuras/rebalance.pdf}}
    \end{center}
    \caption{Geração artificial da classe minoritária para rebalancear a base de imagens. Para cada imagem necessária para igualar o número de imagens da base, $1 \leq n \leq 16$ imagens originais são dadas como entrada para uma operação de geração artificial. A nova imagem é utilizada como treinamento da base.}
  \end{figure}
\end{frame}
%-----------------------------------------------------------------------------
\begin{frame}{Geração de imagens artificiais --- Métodos}
  \setstretch{1.2}
  \setlength\leftmargini{1em}
  \begin{block}{}
    \justifying
    \begin{itemize}
      \item Borramento;
      \item Aguçamento;
      \item Composição;
      \item Mistura ponderada;
      \item Mistura limiarizada;
      \item Mistura saliente;
      \item SMOTE visual;
      \item Adição de ruído.
    \end{itemize}
  \end{block}
\end{frame}
%-----------------------------------------------------------------------------
\begin{frame}{Geração de imagens artificiais --- Borramento}
  \begin{figure}
    \begin{center}
      \subfloat[Original]{
        \includegraphics[width=.48\linewidth]{\detokenize{figuras/artificial-generation/methods/blur-a.png}}
      }
      \subfloat[Imagem artificial]{
        \includegraphics[width=.48\linewidth]{\detokenize{figuras/artificial-generation/methods/blur-b.png}}
      }
    \end{center}
    \caption{Geração artificial utilizando \emph{borramento} com filtro bilateral. A imagem (b) possui detalhes borrados, porém preservando as bordas.}
  \end{figure}
\end{frame}
%-----------------------------------------------------------------------------
\begin{frame}{Geração de imagens artificiais --- Aguçamento}
  \begin{figure}
    \begin{center}
      \subfloat[Original]{
        \includegraphics[width=.48\linewidth]{\detokenize{figuras/artificial-generation/methods/unsharp-a.png}}
      }
      \subfloat[Imagem artificial]{
        \includegraphics[width=.48\linewidth]{\detokenize{figuras/artificial-generation/methods/unsharp-b.png}}
      }
    \end{center}
    \caption{Geração artificial utilizando \textit{unsharp masking}. A imagem resultante (b) apresenta saliência nas transições de intensidade.}
  \end{figure}
\end{frame}
%-----------------------------------------------------------------------------
\begin{frame}{Geração de imagens artificiais --- Adição de ruído}
  \begin{figure}[!htbp]
    \begin{center}
      \subfloat[Original]{
        \includegraphics[width=.48\linewidth]{\detokenize{figuras/artificial-generation/methods/noise-a.png}}
      }
      \subfloat[Imagem artificial]{
        \includegraphics[width=.48\linewidth]{\detokenize{figuras/artificial-generation/methods/noise-b.png}}
      }
    \end{center}
    \caption{Geração artificial utilizando \emph{adição de ruído} de Poisson. Regiões claras de (b) apresentam mais ruído que as regiões escuras.}
  \end{figure}
\end{frame}
%-----------------------------------------------------------------------------
\begin{frame}{Geração de imagens artificiais --- SMOTE visual}
  \begin{figure}
    \begin{center}
      \subfloat[Original]{
        \includegraphics[height=4cm,keepaspectratio]{\detokenize{figuras/artificial-generation/methods/smote-a.png}}
      }
      \subfloat[Original]{
        \includegraphics[height=4cm,keepaspectratio]{\detokenize{figuras/artificial-generation/methods/smote-b.png}}
      }
      \subfloat[Imagem artificial]{
        \includegraphics[height=4cm,keepaspectratio]{\detokenize{figuras/artificial-generation/methods/smote-c.png}}
      }
    \end{center}
    \caption{Geração artificial utilizando o método SMOTE no espaço visual. É possível notar a sobreposição de uma ``sombra'' da Figura (b) em (a).}
  \end{figure}
\end{frame}
%-----------------------------------------------------------------------------
\begin{frame}{Geração de imagens artificiais --- Mistura ponderada}
  \begin{figure}
    \begin{center}
      \subfloat[Original]{
        \includegraphics[width=.32\linewidth]{\detokenize{figuras/artificial-generation/methods/blend-a.png}}
      }
      \subfloat[Original]{
        \includegraphics[width=.32\linewidth]{\detokenize{figuras/artificial-generation/methods/blend-b.png}}
      }
      \subfloat[Imagem artificial]{
        \includegraphics[width=.32\linewidth]{\detokenize{figuras/artificial-generation/methods/blend-c.png}}
      }
    \end{center}
    \caption{Geração artificial utilizando uma \emph{mistura ponderada} de duas imagens. A imagem (c) representa a mistura de (a) e (b).}
    \label{fig:gen:blend}
  \end{figure}
\end{frame}
%-----------------------------------------------------------------------------
\begin{frame}{Geração de imagens artificiais --- Mistura limiarizada}
  \begin{figure}
    \begin{center}
      \subfloat[Original]{
        \includegraphics[width=.32\linewidth]{\detokenize{figuras/artificial-generation/methods/threshold-a.png}}
      }
      \subfloat[Original]{
        \includegraphics[width=.32\linewidth]{\detokenize{figuras/artificial-generation/methods/threshold-b.png}}
      }
      \subfloat[Imagem artificial]{
        \includegraphics[width=.32\linewidth]{\detokenize{figuras/artificial-generation/methods/threshold-c.png}}
      }
    \end{center}
    \caption{Geração artificial utilizando uma \emph{mistura limiarizada} de duas imagens. A imagem resultante (c) é uma composição do \textit{foreground} da primeira imagem sobre o \textit{background} da segunda.}
  \end{figure}
\end{frame}
%-----------------------------------------------------------------------------
\begin{frame}{Geração de imagens artificiais --- Mistura saliente}
  \begin{figure}
    \begin{center}
      \subfloat[Original]{
        \includegraphics[width=.32\linewidth]{\detokenize{figuras/artificial-generation/methods/saliency-a.png}}
      }
      \subfloat[Original]{
        \includegraphics[width=.32\linewidth]{\detokenize{figuras/artificial-generation/methods/saliency-b.png}}
      }
      \subfloat[Imagem artificial]{
        \includegraphics[width=.32\linewidth]{\detokenize{figuras/artificial-generation/methods/saliency-c.png}}
      }
    \end{center}
    \caption{Geração artificial utilizando a \emph{mistura saliente} de duas imagens. A imagem resultante (c) apresenta a região saliente de (b) sobreposta em (a).}
  \end{figure}
\end{frame}
%-----------------------------------------------------------------------------
\begin{frame}{Geração de imagens artificiais --- Composição}
  \begin{figure}
    \begin{center}
      \centering
      \includegraphics[width=0.6\linewidth]{\detokenize{figuras/artificial-generation/methods/composition.png}}
    \end{center}
    \caption{Geração artificial utilizando uma \emph{composição} de imagens. Várias imagens, dispostas em um mosaico, formam a imagem resultante. Cada célula do mosaico sofre uma operação, sorteada no momento da geração da imagem.}
  \end{figure}
\end{frame}
%-----------------------------------------------------------------------------
\subsection{Experimentos}
\begin{frame}{Experimentos}
\setstretch{1.2}
\setlength\leftmargini{1em}
\begin{block}{}
\justifying
\begin{itemize}
\item Comparar a classificação:
\begin{itemize}
\item Base original;
\item Base realçada pelo método proposto.
\end{itemize}
\item Comparar os métodos:
\begin{itemize}
\item Geração artificial;
\item Técnicas de sobreamostragem disponíveis na literatura, como o SMOTE.
\end{itemize}
\end{itemize}
\end{block}
\end{frame}
%-----------------------------------------------------------------------------
\begin{frame}{Experimentos}
  \begin{figure}[!htbp]
  \centering
  \includegraphics[scale=1.1]{\detokenize{figuras/flow_main.pdf}}
  \caption{Fluxo de operações para obtenção dos resultados do rebalanceamento de classes. O mesmo protocolo de conversão para escala de cinza, extração de características e classificação foi seguido para três sub-experimentos: base desbalanceada; base rebalanceada com interpolação dos vetores de características (método SMOTE); e base rebalanceada com a geração artificial de imagens.}
  \end{figure}
\end{frame}
%-----------------------------------------------------------------------------
\begin{frame}{Experimentos}
  \begin{figure}
  \centering
  \includegraphics[scale=0.28]{\detokenize{figuras/folds_chart.png}}
  \caption{Ilustração de como os experimentos de geração de imagens artificiais foram realizados. Primeiramente as imagens são separadas de forma aleatória em $k = 5$ \textit{folds} em cada classe. Em seguida, as duas classes compõem 40 configurações, consistindo em todas as possibilidades de: um fold para teste e os outros como treino para a classe que permanecerá balanceada; e um de teste e apenas um de treino para a classe que os métodos de processamento irão rebalancear. Tal validação é repetida para todas as classes, ou seja, cada classe tem a possibilidade de ser a minoritária.}
  \end{figure}
\end{frame}
% %-----------------------------------------------------------------------------
% \begin{frame}{Avaliação da Classificação}
% % \setstretch{1.2}
% \setlength\leftmargini{1em}
% \begin{block}{Medida F1}
% \justifying
% Problema da acurácia: minoritária sem resultados corretos. \\
% % Performance da classificação em cenários desbalanceados.
% \begin{itemize}
%
% \item Precisão (exatidão): dos exemplos classificados como positivos, quantos realmente são.
% \vspace{-1.5em}
% \begin{equation*}
%   P = \frac{VP}{VP + FP}
% \end{equation*}
% \item Revocação (completude): exemplos positivos corretamente classificados como tal.
% \begin{equation*}
%   R = \frac{VP}{VP + FN}
% \end{equation*}
%
% \pause
% \begin{equation*}
%   F1 = 2 \frac{P \cdot R}{P+R}
% \end{equation*}
% \end{itemize}
% \end{block}
% \end{frame}
%%%%%%%%%%%%%%%%%%%%%%%%%%%%%%%%%%%%%%%%%%%%%%%%%%%%%%%%%%%%%%%%%%%%%%%%%%%%%%%
\begin{frame}{Experimento: duas classes discriminadas}
  \begin{figure}
    \begin{center}
      \subfloat{
        \includegraphics[width=0.48\linewidth]{\detokenize{figuras/corel_original4.jpg}}
      }
      \subfloat{
        \includegraphics[width=0.48\linewidth]{\detokenize{figuras/cavalo-original2.png}}
      }
      \caption{Classes \emph{Horse} e \emph{Elephant} utilizadas no experimento. São duas classes bem discriminadas com 100 imagens cada, originalmente da base de imagens Corel-1000.}
    \end{center}
  \end{figure}
\end{frame}
%-----------------------------------------------------------------------------
\begin{frame}{Protocolo}
  \begin{enumerate}
  \item \textbf{Imagens originais}: classes \emph{Horse} e \emph{Elephant} da base de imagens Corel-1000 \cite{Wang2001}. A principal característica dessas imagens é a diferença de cores, contendo pequeno grau de sobreposição.

  \item \textbf{Desbalanceamento}: para o sub-experimento de visualização, cada classe foi dividida em 50\% para treino e 50\% para teste, de maneira aleatória. Após, a classe \emph{Horse} sofreu remoção de 50\% do seu conjunto de treino, tornando-a desbalanceada. Já para a análise estatística do experimento, todas as 40 configurações de \textit{folds} com $k=5$ foram realizadas (padronização anteriormente descrita na Figura~\ref{fig:folds}).

  \item \textbf{Método para geração artificial}: para a visualização do espaço de características foi utilizado o método de \emph{mistura} de duas imagens originais, exemplificado na Figura~\ref{fig:mistura}. Para a análise do \textit{boxplot} de \textit{F1-Scores}, todas as gerações foram testadas e os resultados são reportados a seguir.

  \item \textbf{Conversão em escala de cinza}: método \emph{Intensidade'} para a visualização. Todas as combinações de extração e conversão em escala de cinza foram testadas, portanto todos os métodos de conversão foram utilizados.

  \item \textbf{Extração de características}: classificação de pixels de borda e interior (BIC) para a visualização. Todos os métodos de extração foram testados para a análise estatística.

  \item \textbf{Classificação}: o classificador supervisionado KNN com $K=1$ (para mais detalhes ver Seção~\ref{sec:knn}) foi utilizado.

  \item \textbf{Projeção multidimensional para visualização}: dois componentes principais encontrados ao aplicar PCA (Seção~\ref{sec:pca}) nos vetores de características para redução de dimensionalidade foram projetados.
  \end{enumerate}
\end{frame}
%-----------------------------------------------------------------------------
\begin{frame}{Descrição do Experimento}
  \begin{figure}
    \begin{center}
      \subfloat[Original]{
        \includegraphics[width=.32\linewidth]{\detokenize{figuras/cavalo-original.png}}
      }
      \subfloat[Original]{
        \includegraphics[width=.32\linewidth]{\detokenize{figuras/cavalo-original2.png}}
      }
      \subfloat[Imagem artificial]{
        \includegraphics[width=.32\linewidth]{\detokenize{figuras/cavalo-blend.png}}
      }
      \caption{Exemplo da geração artificial de imagens com o método de \emph{mistura} para as classes \emph{Elephant} e \emph{Horse} da base Corel-1000. A imagem resultante (c) é composta pela mistura de (a) e (b).}
  \end{center}
  \end{figure}
\end{frame}
%-----------------------------------------------------------------------------
\begin{frame}{Experimento}
  \begin{figure}
    \begin{center}
      \subfloat[Original]{
        \includegraphics[width=.48\linewidth]{\detokenize{figuras/visualizacao/original.png}}
      }
      \subfloat[Desbalanceado]{
        \includegraphics[width=.48\linewidth]{\detokenize{figuras/visualizacao/desbalanceado-fixed.png}}
      }
  \end{center}
  \caption{À esquerda a projeção dos dois componentes principais obtidos com a aplicação de PCA nas classes \emph{Elephant} --- em azul --- e \emph{Horse} --- em verde. À direita, as mesmas classes após a remoção de 50\% das imagens de treino da classe \emph{Horse}. A diferença dos marcadores consiste na definição de imagens para treino e teste não existente nas classes originais.}
  \end{figure}
\end{frame}
%-----------------------------------------------------------------------------
\begin{frame}{Experimento}
  \begin{figure}
    \begin{center}
      \subfloat[SMOTE]{
        \includegraphics[width=.5\linewidth]{\detokenize{figuras/visualizacao/smote-treino-fixed.png}}
      }
      \subfloat[Geração artificial de imagens]{
        \includegraphics[width=.5\linewidth]{\detokenize{figuras/visualizacao/geracao-treino-fixed.png}}
      }
    \end{center}
    \caption{Comparação dos exemplos de treinamento da geração com SMOTE e no campo visual. Em laranja estão representados os novos exemplos, projetados no plano da base original balanceada.}
  \end{figure}
\end{frame}
%-----------------------------------------------------------------------------
\begin{frame}{Experimento}
  \begin{figure}
    \begin{center}
      \subfloat[Smote]{
        \includegraphics[width=.5\linewidth]{\detokenize{figuras/visualizacao/smote-teste-fixed.png}}
      }
      \subfloat[Geração artificial]{
        \includegraphics[width=.5\linewidth]{\detokenize{figuras/visualizacao/geracao-teste-fixed.png}}
      }
  \end{center}
  \caption{Resultado do teste da classificação com K-NN com $K = 1$ após o treinamento realizado com as bases rebalanceadas. A cor no interior dos marcadores quadrados representa a classe real dos exemplos e a borda representa a classe predita pelo classificador.}
  \end{figure}
\end{frame}
%-----------------------------------------------------------------------------
\begin{frame}{Experimento}
  \begin{figure}
      \begin{center}
      \subfloat[Desbalanceado]{
        \includegraphics[width=.5\linewidth]{\detokenize{figuras/visualizacao/desbalanceado-teste-region.png}}
      }
      \end{center}
      \subfloat[Smote]{
        \includegraphics[width=.5\linewidth]{\detokenize{figuras/visualizacao/smote-teste-region.png}}
      }
      \subfloat[Geração artificial]{
        \includegraphics[width=.5\linewidth]{\detokenize{figuras/visualizacao/geracao-teste-region.png}}
      }
    \caption{Região de decisão com K-NN (K = 1). Pode ser observado que em ambas técnicas a região da classe minoritária apresenta-se melhor representada. Além disso, é possível verificar que o SMOTE ocasionou uma certa invasão do espaço de características da classe majoritária.}
  \end{figure}
\end{frame}
%-----------------------------------------------------------------------------
\begin{frame}{Experimento}
  \begin{figure}
      \begin{center}
      \subfloat[Desbalanceado]{
        \includegraphics[width=.5\linewidth]{\detokenize{figuras/visualizacao/desbalanceado-treino.png}}
      }
      \end{center}
      \subfloat[Smote]{
        \includegraphics[width=.5\linewidth]{\detokenize{figuras/visualizacao/smote-treino.png}}
      }
      \subfloat[Geração artificial]{
        \includegraphics[width=.5\linewidth]{\detokenize{figuras/visualizacao/geracao-treino.png}}
      }
    \caption{Melhores subespaços encontrados após a geração de novos exemplos para o SMOTE, para a geração artificial de imagens, e após a remoção de imagens para a projeção dos dados desbalanceados. Pode-se notar que a geração de imagens artificiais proporciona a criação de um subespaço que melhor discretiza as classes, quando comparado com SMOTE ou com a base desbalanceada.}
  \end{figure}
\end{frame}
%-----------------------------------------------------------------------------
\begin{frame}{Experimento}
  \begin{figure}
      \begin{center}
        \includegraphics[width=.8\linewidth]{\detokenize{figuras/visualizacao/vis-images.png}}
      \end{center}
    \caption{Visualização do impacto do método de extração de características na separação entre classes. Possível verificar que o BIC utiliza as intensidades como principal representação de uma imagem.}
  \end{figure}
\end{frame}
%-----------------------------------------------------------------------------
\begin{frame}{Experimento}
  \begin{figure}
    \begin{center}
        \includegraphics[width=\linewidth]{\detokenize{figuras/artificial-generation/1/ACC_Gleam_elefante-cavalo.png}}
    \end{center}
    \caption{Conversão em escala de cinza com \emph{Gleam} e ACC como método de extração de características. Nota-se que o método de geração baseado em Composição 4 obteve maior valor de \textit{F1-Score}.}
  \end{figure}
\end{frame}
%-----------------------------------------------------------------------------
\begin{frame}{Experimento}
  \begin{table}
  \begin{center}
  \caption{Resultados de \textit{F1-Score} para as classes \emph{Horse} e \emph{Elephant}, utilizando \emph{Gleam} como método para conversão em escala de cinza e ACC para extração de características. Nota-se que o método de geração baseado em Composição 4 obteve maior valor de \textit{F1-Score}.}
  \begin{tabular}{|l|c|c|}
  \hline
  \textbf{\emph{Gleam} \& ACC} & \textbf{Média}     & \textbf{Desvio Padrão} \\ \hline
  Todos                 & 91.090913          & 4.559066               \\ \hline
  Aguçamento            & 91.002678          & 4.907016               \\ \hline
  Borramento            & 89.394500          & 5.103498               \\ \hline
  Composição 16         & 90.934305          & 4.399334               \\ \hline
  Composição 4          & \textbf{91.773528} & 4.909852               \\ \hline
  Limiares              & 90.893133          & 5.285833               \\ \hline
  Mistura               & 90.177055          & 4.409787               \\ \hline
  Ruído                 & 89.337770          & 5.169757               \\ \hline
  SMOTE Visual          & 88.616535          & 5.567976               \\ \hline
  Saliência             & 91.282655          & 4.230281               \\ \hline
  SMOTE                 & 90.173808          & 4.566863               \\ \hline
  Desbalanceado         & 88.258567          & 5.538461               \\ \hline
  \end{tabular}
  \end{center}
  \end{table}
\end{frame}
%-----------------------------------------------------------------------------
\begin{frame}{Experimento}
  \begin{figure}
    \begin{center}
      \subfloat[Imagem gerada]{
        \includegraphics[width=.6\linewidth]{\detokenize{figuras/artificial-generation/1/resultado-composicao.png}}
      }
    \end{center}
  \caption{A imagem gerada apresenta uma \emph{composição} de quatro imagens da classe \textit{Elephant}. \\ \textit{Fonte:~Elaborado pela autora.}}
  \end{figure}
\end{frame}
%-----------------------------------------------------------------------------
\begin{frame}{Experimento}
  Para o resultado da combinação dos melhores métodos de conversão em escala de cinza e extração de características, o teste \textit{post-hoc} HSD de Tukey revelou que não há diferença estatística entre a base desbalanceada e o SMOTE ($\textit{p-value} = 0.2073$). Mas indicou que existe uma significância entre o desbalanceamento e a geração artificial ($\textit{p-value} = 0.0062$). Isso significa que o melhor método para rebalancear essas classes é a geração artificial utilizando o método de \emph{mistura}s de duas imagens. Ainda de acordo com o teste, não há evidência estatística da relevância do resultado da combinação de maior variância. Portanto, todos os próximos experimentos relatam apenas os resultados da melhor combinação.
\end{frame}
%-----------------------------------------------------------------------------
\begin{frame}{Experimento}
  \begin{table}[!htbp]
  \centering
  \caption{\textit{Ranking} dos métodos de rebalanceamento ao acumular os resultados de todos os experimentos. Esse valor é dado pela soma da posição de cada método em relação ao \textit{F1-Score}, em ordem ascendente.}
  \label{tab:allranking}
  \begin{tabular}{|l|l|l|}
  \hline
  \multicolumn{1}{|c|}{\textbf{Cenários de duas classes}} & \multicolumn{1}{c|}{\textbf{Cenário Multiclasses}} & \multicolumn{1}{c|}{\textbf{Todos}} \\ \hline
  Aguçamento (16)                                         & SMOTE (13)                                         & Limiares (35)                       \\ \hline
  Limiares (17)                                           & Mistura (14)                                       & Mistura (39)                        \\ \hline
  Saliência (22)                                          & Limiares (18)                                      & SMOTE (42)                          \\ \hline
  Todos (25)                                              & Todos (22)                                         & Aguçamento (43)                     \\ \hline
  Mistura (25)                                            & Saliência (24)                                     & Saliência (46)                      \\ \hline
  Composição 4 (29)                                       & Aguçamento (27)                                    & Todos (47)                          \\ \hline
  SMOTE (29)                                              & Composição 4 (28)                                  & Composição 4 (57)                   \\ \hline
  Composição 16 (37)                                      & Ruído (28)                                         & Composição 16 (69)                  \\ \hline
  Borramento (44)                                         & Borramento (29)                                    & Borramento (73)                     \\ \hline
  Ruído (47)                                              & Composição 16 (32)                                 & Ruído (75)                          \\ \hline
  SMOTE Visual (47)                                       & Desbalanceado (34)                                 & Desbalanceado (86)                  \\ \hline
  Desbalanceado (52)                                      & SMOTE Visual (42)                                  & SMOTE Visual (89)                   \\ \hline
  \end{tabular}
  \end{table}
\end{frame}
%-----------------------------------------------------------------------------
\begin{frame}{Experimento}
  \begin{table}[!htbp]
  \centering
  \caption{Apresenta a média dos \textit{F1-Scores} para cada método, ordenada pela coluna de todos os experimentos. É possível verificar que, no cenário multiclasses, apesar de ter sido o melhor método em relação à sua posição, o SMOTE apresentou piores resultados mesmo comparando com as bases desbalanceadas.}
  \label{tab:allfscore}
  \begin{tabular}{|l|c|c|c|}
  \hline
  \multicolumn{1}{|c|}{\textbf{Métodos}} & \textbf{Cenários de duas classes} & \textbf{Cenário Multiclasses} & \textbf{Todos} \\ \hline
  Aguçamento                             & 84,473575                         & 74,817654                     & 80,182055      \\ \hline
  Limiares                               & 84,332408                         & 74,472474                     & 79,950215      \\ \hline
  Saliência                              & 84,172238                         & 74,248645                     & 79,761752      \\ \hline
  Composição 4                           & 83,123738                         & 74,667785                     & 79,365537      \\ \hline
  Composição 16                          & 82,977850                         & 74,581912                     & 79,246322      \\ \hline
  Mistura                                & 83,124582                         & 74,045887                     & 79,089606      \\ \hline
  Borramento                             & 82,164793                         & 74,555320                     & 78,782805      \\ \hline
  Desbalanceado                          & 81,314335                         & 74,551506                     & 78,308633      \\ \hline
  Todos                                  & 82,358089                         & 71,943914                     & 77,729567      \\ \hline
  Ruído                                  & 81,179247                         & 72,609639                     & 77,370532      \\ \hline
  SMOTE                                  & 79,085501                         & 73,329079                     & 76,527091      \\ \hline
  SMOTE Visual                           & 78,811666                         & 70,439604                     & 75,090749      \\ \hline
  \end{tabular}
  \end{table}
\end{frame}
%%%%%%%%%%%%%%%%%%%%%%%%%%%%%%%%%%%%%%%%%%%%%%%%%%%%%%%%%%%%%%%%%%%%%%%%%%%%%%%
\begin{frame}{Resultados - Rede de Convolução}
\begin{table}
\url{http://caffe.berkeleyvision.org/}
\caption{Treinamento das classes praia e montanha da base COREL-1000.}
  \begin{tabular}{c|c}
    Bases    &   Medida F1 \\ \hline
    Original não balanceada     &   0.708  \\
    Desbalanceada em 50\% &   0.577  \\
    Rebalanceada  &   0.677  \\
  \end{tabular}
\end{table}
\end{frame}
%%%%%%%%%%%%%%%%%%%%%%%%%%%%%%%%%%%%%%%%%%%%%%%%%%%%%%%%%%%%%%%%%%%%%%%%%%%%%%%
\begin{frame}{Artigo publicado na Neurocomputing}

\begin{columns}
  \begin{column}{0.5\textwidth}
  \centering
\begin{block}{}
\justifying
\tiny{
Ponti, M.; Nazaré, T; Thumé, G. \textbf{Image quantization as a dimensionality reduction procedure in color and texture feature extraction}, submitted to Neurocomputing, 2014.}
\end{block}
  \end{column}
  \begin{column}{0.5\textwidth}
  \centering
    \includegraphics[width=0.9\linewidth]{figuras/artigo.png}
  \end{column}
\end{columns}
\end{frame}
%-----------------------------------------------------------------------------
\begin{frame}{Trabalhos futuros}
\setlength\leftmargini{0em}
\setstretch{1.2}
\justifying
  \begin{itemize}
  \item Analisar a memória associativa aprendida com uma máquina de Boltzmann restrita.
  \begin{itemize}
    \item Escolher para qual imagem original utilizar, ao invés do método aleatório utilizado nos resultados preliminares;
    \item Verificação da relevância das imagens geradas.
  \end{itemize}
  \end{itemize}
\end{frame}
%%%%%%%%%%%%%%%%%%%%%%%%%%%%%%%%%%%%%%%%%%%%%%%%%%%%%%%%%%%%%%%%%%%%%%%%%%%%%%%
\begin{frame}{Agradecimentos}
  Moacir Antonelli Ponti \\
  João do Espirito Santo Batista Neto \\

\centering{
  \begin{columns}
    \begin{column}{0.6\textwidth}
      \begin{figure}[!htbp]
        \begin{center}
        \subfloat{
          \includegraphics[width=0.6\columnwidth]{figuras/cnpqLogo.jpg}
        }
        \newline
        \subfloat{
          \includegraphics[width=0.4\columnwidth]{figuras/capesLogo.png}
        }
        \end{center}
      \end{figure}
    \end{column}
    \begin{column}{0.3\textwidth}
      \includegraphics[width=\columnwidth]{figuras/brasao_usp_pb}
    \end{column}
  \end{columns}
}
\end{frame}
%-----------------------------------------------------------------------------
% \section{Referências}
\nocite{*}
%\section{Referências}
\begin{frame}[allowframebreaks]{Referências}
\tiny
% \bibliographystyle{apacite}
\bibliographystyle{plain}
%\bibliographystyle{amsalpha}
\bibliography{referencias}
\end{frame}

%-----------------------------------------------------------------------------
\begin{frame}[plain]
  \maketitle
\end{frame}
%%%%%%%%%%%%%%%%%%%%%%%%%%%%%%%%%%%%%%%%%%%%%%%%%%%%%%%%%%%%%%%%%%%%%%%%%%%%%%%%
\end{document}
